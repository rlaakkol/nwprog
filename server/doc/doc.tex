\documentclass[a4paper,12pt]{article}
\usepackage[utf8x]{inputenc}
\usepackage{graphicx}
\usepackage{hyperref}

\hypersetup{
    bookmarks=true,         % show bookmarks bar?
    unicode=false,          % non-Latin characters in Acrobat’s bookmarks
    pdftoolbar=true,        % show Acrobat’s toolbar?
    pdfmenubar=true,        % show Acrobat’s menu?
    pdffitwindow=false,     % window fit to page when opened
    pdfstartview={FitH},    % fits the width of the page to the window
    pdftitle={Documentation},    % title
    pdfauthor={RL},     % author
    pdfsubject={Subject},   % subject of the document
    pdfcreator={Creator},   % creator of the document
    pdfproducer={Producer}, % producer of the document
    pdfkeywords={keywords}, % list of keywords
    pdfnewwindow=true,      % links in new window
    colorlinks=true,       % false: boxed links; true: colored links
    linkcolor=black,          % color of internal links
    citecolor=red,        % color of links to bibliography
    filecolor=magenta,      % color of file links
    urlcolor=blue           % color of external links
}

\usepackage[School of Electrical Engineering]{aaltologo}
\usepackage{parskip}

\begin{document}

\begin{titlepage}%
    \begin{flushleft}
	    \AaltoLogoRandomSmall{1}\\
S-38.3610 -- Network Programming\\
Spring 2013\\
\end{flushleft}


    \vspace*{\stretch{1}}
    \begin{center}
	    \LARGE{\textbf{HDserver}\\
	    \large{Documentation}
    \end{center}
    \vspace*{\stretch{1}}

    \begin{center}%
      \begin{tabular}{ll}
        Riku Lääkkölä & 69896S \\ 
		riku.laakkola@aalto.fi & \\
        \end{tabular}

    \end{center}
  \end{titlepage}

\tableofcontents

\section{Usage Instructions and Disclaimer}

\emph{DISCLAIMER 1:} For this assingment I had two attempts. First, I tried to implement a single-threaded server that uses select() and nonblocking I/O for multiplexing. I managed to get serving GET request to be somewhat functional, but otherwise there seem to be problems. I had to fall back to using code that I wrote for last year's course that forks a child process for each client, which is somewhat more inefficient, but at least the code works...

And now for the instructions: First, the program must be compiled by running {\tt make}. There are no special requirements (at least make runs flawlessly on the course servers) After this, just run the server! Usage as follows:

To start child forking server:

{\tt ./fork_srv <working_directory> <port>}

To start selecting server:

{\tt ./select_srv [-d] [-p <working_directory>] <port>} (\emph{d} for daemonize)

The forking program logs information to LOG_LOCAL7 and the selecting one prints very verbosely to stdout (unless run as daemon).

\section{Overview}

The program serves HTTP GET and PUT requests from clients. The program can handle both IPv4 and IPv6 addresses, and when using DNS names, the user does not need to worry about the IP version.

\subsection{Forking version}
Internally, the software is divided into four components which are:

\begin{enumerate}
\item \emph{myhttp}, which generates and parses HTTP messages and reads and writes them from sockets.
\item \emph{npbserver}, which handels DNS queries, establishing connection to the server and separating the child logic from the main loop.
\item \emph{main}, which parses the user provided command line arguments, daemonizes the process and handles the main accept loop and forking
\end{enumerate}

More specifically, the program handles starting itself and processing a GET request from a client in the following manner:

\sloppy

\begin{enumerate}
\item The listening socket is opened using tcp\_listen in npbserver.
\item A for-loop is started. The loop blocks at accept until a client connection attempt arrives.
\item If connection is established successfully, a child is forked, and functionality moves to the web\_child function in npbserver.c.
\item Data is read from the socket to a buffer until CRLF occurs, and after that, the header is parsed.
\item If the header is valid, the appropriate function for the request is called.
\item The process\_get function first writes the appropriate header to the socket and then starts reading the file into a buffer and writing to the socket from the buffer.
\end{enumerate}

\fussy


\emph{DISCLAIMER 2:} The \emph{readn} and \emph{writen} functions which try to write and read exactly n bytes from or into a socket are copied almost directly from the Stevens book examples (also available in lecture materials in Optima). Also the \emph{tcp\_connect}, \emph{daemon\_init} and \emph{tcp\_listen} function is copied almost in full from the lecture examples.

\subsection{Selecting version}

The selecting version tries to handle parallel activity using select and nonblocking I/O. The listening socket is set up in the same way as above, but instead of a blocking accept in the main loop, we have select that returns whenever there are events in sockets we are interested in.

In the beginning the only socket that is set in the fd\_sets provided to select is the listening socket. When select returns, it means that there is a connection attempt. If the connection is successfull, a new client structure is initialized and appended to a linked list of all clients. This list is checked always before select, and relevant clients are added to relevant fd\_sets.

When select returns, the list of clients is checked again in order to see if their respective sockets have become readable or writable during select. If so, depending on the socket status and client state, the appropriate functions are called for processing requests. The processing is fairly similar to the forking version, with the exception that each I/O function is called only once, and if they fail with EWOULDBLOCK, the function returns to the select loop. Also the client struct pointer is passed along, so that the client state may be altered by the functions.

\section{Testing and limitations}

The servers were tested on the nwprog test servers and my home computer, using the client from phase 1 and normal web browsers. The selecting version does not handle unexpected conditions nicely... The forking version is a bit better.

\section{Specifics}

\begin{enumerate}
\item The forking version uses separate processes for parallel activity, and the selecting version uses select and nonblocking I/O. If a connected client stops transfer, the I/O functions will block until the default timeout and after that the child exits. The selecting one will have the bad client in the list forever, but the sockets will not appear active in the fd\_sets.
\item The implementation is address family independent. This is accomplished by providing \emph{getaddrinfo} with a \emph{hints} structure that has \emph{ai\_family} of value \emph{AF\_UNSPEC}. This was tested by running the software from my home (IPv6 connectivity) and Otakaari 5 (no IPv6) and some informative statistics printing.
\item For some reason, apache benchmark timeouts when stress-testing, so no acceptable high load test results are available. The correct number of bytes seem to be transferred with tens of simultaneous connections with ab. With keepalive enabled, 10 concurrent connections go through, but 100 kills the server.
\end{enumerate}

\section{Diary}



\end{document}